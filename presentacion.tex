\documentclass{beamer}
%\usepackage[latin1]{inputenc}
\usepackage[spanish]{babel}
\usepackage[T1]{fontenc}
\usepackage{graphicx,hyperref,ru1,url}
%\usefonttheme{serif}
%\setmainfont{Linux Libertine O}

\usepackage{subfiles}
\usepackage{subfig} 
\usepackage{pdfpages}
\usepackage{amsthm}

\usepackage{tikz}
%\RequirePackage{tikz-uml}
%\RequirePackage{pgf-umlcd}
\usetikzlibrary{positioning,decorations.pathreplacing,shapes}
\usetikzlibrary{positioning,fit,babel,shapes.multipart,calc}
\usetikzlibrary{arrows.meta,bending}
\usetikzlibrary{external}




% The title of the presentation:
%  - first a short version which is visible at the bottom of each slide;
%  - second the full title shown on the title slide;
\title[Segmentación de imágenes con lógica difusa]{
  Segmentación de imágenes a través de lógica difusa y funciones REF, de Dombi y penalti}

% Optional: a subtitle to be dispalyed on the title slide
\subtitle{}

% The author(s) of the presentation:
%  - again first a short version to be displayed at the bottom;
%  - next the full list of authors, which may include contact information;
\author[Iñigo Aguas Ardaiz]{
  I\~nigo Aguas Ardaiz \\\medskip\\
  {\small Trabajo de Fin de Grado}}

% The institute:
%  - to start the name of the university as displayed on the top of each slide
%    this can be adjusted such that you can also create a Dutch version
%  - next the institute information as displayed on the title slide
\institute[Universidad P\'ublica de Navarra]{
Grado en Ingeniería Informática\\
E.T.S. de Ing. Industrial, Informática y de Telecomunicación\\
Universidad Pública de Navarra}

% Add a date and possibly the name of the event to the slides
%  - again first a short version to be shown at the bottom of each slide
%  - second the full date and event name for the title slide
\date[25 de junio de 2015]{
25 de junio de 2015}

\begin{document}

\begin{frame}
  \titlepage
\end{frame}

\begin{frame}
  \frametitle{Outline}

  \tableofcontents
\end{frame}

% Section titles are shown in at the top of the slides with the current section 
% highlighted. Note that the number of sections determines the size of the top 
% bar, and hence the university name and logo. If you do not add any sections 
% they will not be visible.
\section{Introducción}

\begin{frame}
  \frametitle{Introduction}

  \begin{itemize}
    \item This is just a short example
    \item The comments in the \LaTeX\ file are most important
    \item This is just the result after running pdflatex
    \item The style is based on the webpage \url{http://www.ru.nl/}
  \end{itemize}
\end{frame}

\section{Conceptos báiscos}

\begin{frame}
  \frametitle{Background information}

  \begin{block}{Slides with \LaTeX}
    Beamer offers a lot of functions to create nice slides using \LaTeX.
  \end{block}

  \begin{block}{The basis}
    This style uses the following default styles:
    \begin{itemize}
      \item split
      \item whale
      \item rounded
      \item orchid
    \end{itemize}
  \end{block}
\end{frame}

\section{Dombi}

\begin{frame}
  \frametitle{The important things}

  \begin{enumerate}
    \item This just shows the effect of the style
    \item It is not a Beamer tutorial
    \item Read the Beamer manual for more help
    \item Contact me only concerning the style file
  \end{enumerate}
\end{frame}

\section{OWA}

\begin{frame}
  \frametitle{OWA}

sjkdnfjksndfkjnsdkjfnjksnfkjnsdkjnfkjsnkjfnskjdnkjfnskjnfkjsd

  This style file gives your slides some nice Radboud branding.
  When you know how to work with the Beamer package it is easy to use.
  Just add:\\ ~~~$\backslash$usepackage$\{$ru$\}$ \\ at the top of your file.
\end{frame}

\section{Conclusiones y líneas de futuro}{

\begin{frame}
  \frametitle{Conclusion}

  \begin{itemize}
    \item Easy to use
    \item Good results
  \end{itemize}
\end{frame}
}

\begin{frame}
\begin{block}{Teorema 1} Teorema 1
\end{block}
\begin{exampleblock}{Ejemplo 1} Ejemplo 1
\end{exampleblock}
\begin{alertblock}{Observación 1} Observación 1
\end{alertblock}
\begin{block}{} Teorema 1 \end{block}
\begin{exampleblock}{} Ejemplo 1 \end{exampleblock}
\begin{alertblock}{} Observación 1 \end{alertblock}
\end{frame}

\begin{frame}
  \subfile{graficos/logicaclasicagraf.tex}
\end{frame}

\begin{frame}
  \titlepage
\end{frame}

\end{document}
