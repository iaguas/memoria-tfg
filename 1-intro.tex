% INTRODUCCIÓN
\chapter{Introducción}

En este capítulo se da una visión general del problema que se ha estudiado así como la motivación para investigar y contribuir en este área. Se explica, también, la relevancia del problema en el marco del procesamiento de imagen y la segmentación. Finalmente, se especifican el propósito y los objetivos que se han guiado esta investigación


% 1.1. MOTIVACIÓN
\section{Motivación}\label{sec:motivacion}

El desarrollo de mecanismos que permiten a las máquinas el aprendizaje de técnicas que les permitan la resolución automática de problemas es un campo fundamental dentro del área de la Computación y la Inteligencia Artificial.Este tipo de procesos se utilizan en la vida diaria de muchos seres humanos y resultan innatos en ellos, en cambio, su implementación para que las máquinas los implementen se convierte en todo un reto debido a la dificultad de la imitación de los procesos cerebrales de las personas.

Muchos autores \cite{lib:ross, lib:boden, art:searle, art:churchland} han propuesto ideas sobre esta cuestión pero debe destacarse a R. Penrose \cite{lib:penrose} que en su libro {\em La nueva mente del Emperador},  se proclama un claro detractor de la idea de que las máquinas puedan llegar a tener la opción de discernir de una forma similar al cerebro humano. Llega incluso a preguntarse ``?`cómo podríamos siquiera {\em empezar} a explicar la substancia de tales problemas a una entidad que no sea ella misma consciente\dots?''

En definitiva, en este trabajo se motiva en la idea de hacer posible lo que muchos creen imposible, lo que muchos creen que no será posible en mucho tiempo, incluso que no será posible sin conocernos a nosotros mismos. En este sentido, se trata el problema de la segmentación de imágenes para intentar mejorar los métodos actuales y llegar a un método que pudiera ser utilizado de forma general para cualquier entrada.


% 1.2. DEFINICIÓN
\section{Definición del problema}\label{sec:definicion}
El problema de la segmentación consiste en poder conocer dentro de una imagen qué parte de ella pertenece al objeto y cual al fondo. Así, por ejemplo, en la figura \ref{img:ejemplomolino} el lector podrá diferenciar claramente un molino del pueblo que se ve en el fondo de la imagen. El hecho de diferenciar un objeto del fondo de una imagen es un proceso que no supone ninguna dificultad para la mente humana; es algo que realizamos inconscientemente cientos de veces durante el día, sin que nos cueste ningún esfuerzo. Sin embargo, la misma tarea puede resultar muy complicada o imposible de realizar a través de un programa informático. Podría ser como aquel hidalgo que veía gigantes en lugar molinos.

\begin{figure}
	\centering
	\includegraphics[width=0.9\textwidth]{img/molino.jpg}
	\caption{Distinguir el molino del pueblo del fondo no es difícil para un humano aunque sí para una máquina.}
	\label{img:ejemplomolino}
\end{figure}

\REV{pq es importante el problema de la segmentación} Autores como González y Woods en \cite{lib:gonzalez} enuncian el problema de ``segmentación de imágenes no triviales como una de las tareas más dificiles en el procesamiento de imágenes''. En este sentido, insisten en que ``la exactitud de la segmentación determina el éxito o error de los procesos de análisis computerizados''. Otros autores \cite{lib:sonka} hablan de la sementación como una ``técnica en al que se divide la imagen en partes que tienen una correlación con objetos o áreas del mundo real contenidas en la imagen''. Definido formalmente:

\begin{definition}\label{def:definicionproblema}
Dada una imagen $Q$ que se pude subdividir en $n$ regiones $R_{1}, \dots, R_{n}$, y sabemos que $P$ es una cierta propiedad booleana que cumplen todos los píxeles de la región $R_{i}, \forall  i=1,\dots ,n$, se deberá cumplir siempre que:
\begin{enumerate}
	\item $\bigcup_{i=1}^{n}R_{i}=Q$;
	\item En una región $R_{i}, \forall i=1,\dots ,n$ todos sus píxeles están conectados;
	\item $R_{i}\cap R_{j}=\emptyset, \forall i, j : i\neq j;$
	\item $P(R_{i}) = \text{ VERDADERO }, \forall  i=1,\dots ,n;$
	\item $P(R_{i}\cup R_{j}) = \text{ FALSO }, \forall  i=1,\dots ,n.$
\end{enumerate}
\end{definition}

En conclusión, en este proceso se lleva a cabo la división de la imagen en regiones donde cada una (desde 2 hasta $n$, este número dependerá del problema que estemos resolviendo) harán mención al fondo y a cada objeto. Además, todas las regiones serán independientes entre sí, esto es, un pixel pertenecerá solamente a la región $i$ cuando hablemos de segmentación completa. Centrando el problema únicamente en aquellas imágenes en escala de grises, nuestra pretensión será obtener aquellas regiones que contienen a un objeto centrándonos en sus tonalidades, esto es, por medio de técnicas de umbralización \REV{seguro?}.


% 1.3. SOLUCIÓN
\section{Solución propuesta}\label{sec:solucion}
\REV{Nosotros vamos a utilizar técnicas difusas, para empezar. ¿Por qué? (Incertidumebre en torno a los píxeles, péridda de información captación imágenes,...)}

Existen tres técnicas para poder llevar a cabo la segmentación de una imagen:
\REV{explicación y referencia}
\begin{enumerate}[label=\alph*)]
	\item Segmentación basada en umbralización.
	\item Técnicas basada en agrupamiento de píxeles en regiones.
	\item Técnicas basadas en la detección de bordes.
\end{enumerate}

En este trabajo se va a centrar la segmentación por medio de las técnicas de umbralización. Para ello lo que tendremos que hacer será calcular los umbrales que separen las regiones que se hayan encontrado, de forma que situaremos las regiones entre un umbral $t_{i}$ y uno $t_{i+1}$. Para ejemplificar esto, tomaremos la umbralización binaria en la cual se dispondrá de un único umbral $t$. De esta forma, todos los elementos de la imagen que se encuentren por encima del umbral pertenecerán a una región  y los que estén por debajo a la segunda. Esta técnica se basa únicamente en detectar los diferentes tonos de gris de la imagen, así que mirando el histograma de la imagen (fig. \ref{img:rice}, podríamos ver cómo existe una frontera entre la intensidad $t=71$ y el resto creando diferencias en los niveles de gris. \cite{art:refbarrenechea}

\begin{figure}
\centering
	\subfloat{\includegraphics[width=0.3\textwidth]{img/rice}}
	\subfloat{\includegraphics[width=0.3\textwidth]{img/umbra-rice}}
	\subfloat{\includegraphics[width=0.3\textwidth]{img/hist-rice}}
	\caption{(a) Imagen original. (b) Imagen segmentada. (c) Histograma de la imagen original. Se puede ver que en t=71 se produce un mínimo con respecto a los otros niveles de gris y que por eso se escoge como umbral.}
	\label{img:rice}
\end{figure}


\REV{solución al pq...} Se debe tener en cuenta que la umbralización es un método rápido y de coste computacional bajo por lo que se puede realizar en tiempo real. Como sólo basa su algorítmica en el histograma de la imagen hace que sea una método sencillo e intuitivo aunque esto también hace que tenga problemas ante el ruido y objetos o fondos que no sean uniformes.

La solución que se presenta en este trabajor se haya por medio de lógica difusa. Se han utilizado funciones REF, de Dombi, de agregación y penalti. En el capítulo \ref{basicos} se hace mención a todos estos conceptos y se explican poníendose en práctica a partir del capítulo \ref{monoumbral}. En el capítulo \ref{cap:conclusiones} se presentan todas las conclusiones obtenidas así como las líneas de trabajo futuro.

%La umbralización es rápida, de coste computacional bajo y se puede realizar en tiempo real.
%Sencilla e intuitiva.
%Técnica útili si el fondo y los objetos son uniformes.
%Problemas: el ruido, niveles de gris similares entre obejto y fondo, solapamientos de objetos...
%La obtención de un umbral se basará en el histograma de la imagen. No se considera la información espacial.

% 1.4. REELEVANCIA
\section{Reelevancia}\label{sec:reelevancia}

En el campo de la medicina \cite{lib:suri} se ha experimentado una mejora sustancial en la efectividad de las pruebas médicas gracias a diferentes opciones como los rayos X, la tomografía computerizada, la resonancia magnética, la tomografía por emisión de positrones (PET), imágenes de ultrasonidos y otras. La revolución digital y el gran poder de procesamiento que disponen los ordenadores ha conseguido que los profesionales comprendan la compleja anatomía humana, aunque esto no ha sido suficiente ya que se ha visto necesario poder obtener los bordes, las superficies y la segmentación de los órganos. Estos órganos segmentados y sus bordes son clave para poder conseguir que un especialista médico pueda hacer una cirugía adecuada para muchas ramas de la medicina, debido a la importancia de tener datos en tiempo real. \REV{[Insertar figura que vaya con esto].}


Pruebas médicas
Localización de tumores y otras patologías
Medida de volúmenes de tejido
Cirugía guiada por ordenador
Diagnóstico
Planificación del tratamiento
Estudio de la estructura anatómica

Análisis automático de detección de errores.

Sensor de huella digital
Localización de objetos en imágenes de satélite (teledetección).
\REV{Alargar esto}


% 1.5. OBJETIVOS
\section{Propósito y objetivos}\label{sec:objetivos}

Este proyecto se centra en estudiar técnicas de segmentación para un único umbral y la generalización de estas en múltiples umbrales en imágenes en blanco y negro e intentar mejorar las técnicas de las que actualmente se disponen intentando generalizarlas de forma que los parámetros no dependan del problema. 

Además, para poder conseguir el propósito anterior se estipularon los siguiente objetivos:
\begin{itemize}
	\item Investigar y conocer técnicas actuales de segmentación de imagen de forma que estas sean el punto de partida.
	\item Analizar y evaluar las funciones que J. Dombi propone en \cite{art:dombi}. Comparar estas con las funciones REF y sustituirlas en la construcción de los conjuntos difusos para conocer sus efectos.
	\item Implementar diferentes algoritmos de segmentación con el conocimiento adquirido anteriormente evaluando su mejora y haciendo las correcciones necesarias con la intención de generalizar el método de forma máxima.
	\item Implementar diferentes algoritmos que incluyan la agregación de las diferentes funciones estudiadas para la segmentación en los puntos anteriores comprobando si esto mejora los resultados anteriores.
	\item Analizar todos los puntos anteriores a fin de concluir los resultados del trabajo así como dirimir si se ha podido conseguir cumplir el propósito inicial.
\end{itemize}

%1.6 ANÁLISIS BIBLIOGRÁFICO
\section{Análisis bibliográfico}

No tengo claro aun que poner aquí. Próximamente.