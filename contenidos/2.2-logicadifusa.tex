\documentclass[main]{subfiles}

\begin{document}

% LOGICA DIFUSA
\section{Lógica difusa}\label{sec:logicadifusa}
La lógica difusa fue introducida por el matemático L. A. Zadeh \cite{art:zadeh} en 1965 con la intención de poder extender la lógica clásica o {\em crips} de forma que permitiera manejar y procesar la información que se compone de términos inexactos, imprecisos o subjetivos. Es, al fin y al cabo, un intento de imitar la forma de deducción del cerebro humano para trasladarlo a las máquinas. Se comenzará recordando la lógica clásica para después extender los conceptos a la difusa.

En primer lugar, definiremos el universo finito, $U$,  con el que se  trabajará, de forma que contenga todos aquellos elementos con los que se desee trabajar, esto es, ${U = \{u_{1}, u_{2}, \dots, u_{n}\}}$. En la lógica clásica simplemente asignamos verdadero o falso a cada uno de los elementos del conjunto que se estudia. Por esta razón, al definir un conjunto $A$ podremos decir que este contiene o no a los elementos del universo $U$ con una certeza absoluta. De esta manera, daremos un valor de 1 cuando ${u_{i}}$ esté incluido en $A$ (verdadero) y un valor de 0 cuando no lo esté (falso). En esta última idea se refleja por medio de la función de pertenencia al conjunto  $A$, $\mu_{A}$ (ecuación \ref{eq:logicaclasica}).

\begin{gather}\label{eq:logicaclasica} % TODO: El entorno este es una mierda.
	A = \{(u_{i}, \mu_{A}(u_{i})) | u_{i}\in U\}\\
	\mu_{A}:U\rightarrow \{0,1\} \text{ tal que }
	\mu_{A}(u) = \left\{
		\begin{aligned}
			1 \quad\text{si}\quad u\in A\\
			0 \quad\text{si}\quad u\notin A
		\end{aligned}
	\right.
\end{gather}
% \REV{revisar ecuación con respecto al ejemplo} Se da más adelante la concreta.

Podemos considerar el problema de determinar si una persona es alta o no. Para la lógica clásica, este razonamiento se simplificará en buscar un valor a partir del cual podamos definir que cierta persona es alta. Para el siguiente ejemplo, tomaremos 1,75 metros como referencia, donde $A$ es el conjunto de las personas altas. De este modo, ${\mu_{A}=1 \text{ si } u>1,75}$ y en otro caso ${\mu_{A}=0}$. En la figura \ref{fig:altoclasica} se puede ver la representación del concepto `alto' para las posibles alturas que se pueden presentar.

%\subfile{graficos/logicaclasicagraf}

Por otra parte, la lógica difusa dispone de la función de pertenencia más compleja, lo que nos hace poder decir que alguien es `poco alto' o `bastante alto' ya que no daremos un par de valores (${\{0,1\}}$) sino cualquiera de los contenidos en el intervalo que definen. Así ${\mu_{A}}$ será una función que asigna un valor entre 0 y 1 a cada elemento de $A$.

\begin{gather}\label{eq:logicadifusa} %TODO: El entorno es una mierda, también.
	A = \{(u_{i}, \mu_{A}(u_{i})) | u_{i}\in U\}\\
	\mu_{A}:U\rightarrow [0,1]
\end{gather}

Si continuamos con el ejemplo se verá que el conjunto alto ($A$) esta vez se define como explica la ecuación \ref{eq:ejemplodifusa}.

\begin{equation}\label{eq:ejemplodifusa}
	\mu_{A}(u) = \left\{
		\begin{aligned}
			1 \quad&\text{si}\quad u\geq 2\\
			2u - 3 \quad&\text{si}\quad 1.5>u>2\\
			0 \quad&\text{si}\quad u\leq 1.5
		\end{aligned}
	\right.
 \end{equation}

Esta nueva función de pertenencia hace que podamos distinguir 3 zonas dentro de su representación (figura \ref{fig:altodifusa}). Se tendrá de nuevo la etiqueta `alto' y `no alto' que hacen que su pertenencia sea certera. Se dispondrá, también, una parte de la pertenencia a la que llamaremos `difuso' donde el conjunto no afirma ser ni `alto' ni `no alto' sino que está en una situación intermedia. En este caso se habla de que el elemento $a_{i}$ que se encuentra ahí pertenece a $A$ con grado ${\mu_{A}(a_{i})}$.

%\subfile{graficos/logicadifusagraf}

Más formalmente, denotaremos al conjunto que incluye a todos los conjuntos difusos como $\FF(u)$ a todos aquellos que se encuentran definidos sobre un referencial finito (${|U| = n}$) por lo que $U$ no será un conjunto vacío.

\end{document}
