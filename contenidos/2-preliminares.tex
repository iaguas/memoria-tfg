\chapter{Conceptos básicos}\label{preliminares}\label{basicos}

Este capítulo pretende ser una introducción a todos los conceptos teóricos necesarios para la correcta comprensión del trabajo que se detalla en esta memoria. 

% LOGICA DIFUSA
\section{Lógica difusa}\label{sec:logicadifusa}
La lógica difusa fue ideada por el matemático L. A. Zadeh \cite{art:zadeh} en 1965 con la intención de poder extender la lógica clásica o {\em crips} de forma que permitiera manejar y procesar la información que se compone de términos inexactos, imprecisos o subjetivos. Es, al fin y al cabo, un intento de imitar la la forma de deducción del cerebro humano para trasladarlo a las máquinas. Se comenzará recordando la lógica clásica para después extender los conceptos a la difusa.

En primer lugar, definiremos el conjunto del universo finito, $U$,  con el que se  trabajará, de forma que se contengan todos aquellos elementos con los que se desee trabajar, esto es, $U = {u_{1}, u_{2}, \dots, u_{n}}$. En la lógica clásica simplemente asignamos verdadero o falso a cada uno de los elementos del conjunto que se estudia. Por esta razón, al definir un conjunto $A$ podremos decir que este contiene o no a los elementos del universo $U$ con una certeza absoluta. De esta manera, daremos un valor de 1 cuando $u_{i}$ esté incluido en $A$ (verdadero) y un valor de 0 cuando no lo esté (falso). En esta última idea se refleja por medio de la función de pertenencia al conjunto  $A$, $\mu_{A}$ (ecuación \ref{eq:logicaclasica}).
\begin{equation}\label{eq:logicaclasica}
\begin{aligned} 
	A = \{(u_{i}, \mu_{A}(u_{i})) | u_{i}\in U\}\\
	\mu_{A}:U\rightarrow \{0,1\} \text{ tal que}\\
	\mu_{A}(u) = \left\{ \begin{aligned}
		1 \quad\text{si}\quad u\in U\\
		0 \quad\text{si}\quad u\notin U
 	\end{aligned}\right.
\end{aligned}
\end{equation}
%\REV{revisar ecuación con respecto al ejemplo} Se da más adelante la concreta.

Se puede plantear el problema que nos dé a conocer si una persona es alta o no. Para la lógica clásica, este razonamiento se simplificará en buscar un valor a partir del cual podamos definir que cierta persona es alta. Para el siguiente ejemplo, tomaremos 1,75 metros como referencia. De este modo, $\mu_{A}=1 \text{ si } u>1,75$ y en otro caso $\mu_{A}=0$. En la figura \ref{fig:altoclasica} se puede ver la representación del concepto `alto' para las posibles alturas que se pueden presentar.

\subfile{graficos/logicaclasicagraf}

Por otra parte, la lógica difusa dispone de la función de pertenencia más compleja, lo que nos hace poder decir que alguien es `poco alto' o `bastante alto' ya que no daremos un par de valores ($\{0,1\}$) sino todos los contenidos en el intervalo que definen. Así $\mu_{A}$ será una función que asigna un valor entre 0 y 1 a cada elemento de $A$.\begin{equation}\label{eq:logicadifusa}
\begin{aligned} 
	A = \{(u_{i}, \mu_{A}(u_{i})) | u_{i}\in U\}\\
	\mu_{A}:U\rightarrow [0,1]
\end{aligned}
\end{equation}                
Si continuamos con el ejemplo se verá que el conjunto alto ($A$) esta vez se define como explica la ecuación \ref{eq:ejemplodifusa}. 

\begin{equation}\label{eq:ejemplodifusa}                
	\mu_{A}(u) = \left\{ \begin{aligned}
		1 \quad&\text{si}\quad u\geq 2\\
		2u + 1,5 \quad&\text{si}\quad 1.5>u>2\\
		0 \quad&\text{si}\quad u\leq 1.5
 	\end{aligned}\right.
 \end{equation}           
Esta nueva función de pertenencia hace que podamos distinguir 3 zonas dentro de su representación (figura \ref{fig:altodifusa}). Se tendrá de nuevo el área `alto' y `no alto' que hacen que su pertenencia sea certera. Se dispondrá, también, una parte de la pertenencia a la que llamaremos 'difuso' donde el conjunto no afirma ser ni `alto' ni `no alto' sino que está en una situación intermedia. En este caso se habla de que el elemento $a_{i}$ que se encuentra ahí pertenece a $A$ con grado $\mu_{A}(a_{i})$.
                
\subfile{graficos/logicadifusagraf}

Más formalmente, denotaremos al conjunto que incluye a todos los conjuntos difusos como $\FF(x)$ a todos aquellos que se encuentran definidos sobre un referencial finito ($|X| = n$) por lo que $X$ no será un conjunto vacío.
 

% FUNCIONES REF
\section{Funciones de equivalencia restringida}\label{sec:ref}
El concepto de función de equivalencia restringida (REF por sus siglas en inglés) surge del concepto de equivalencia y el de similitud \cite{art:refbarrenechea}. Este concepto es muy utilizado para la comparación de imágenes y su intención es dar una medida de cómo de iguales o similares son dos elementos $x$ e $y$. Para poder definir el concepto de REF, necesitamos varios previamente. 

\begin{definition}\label{def:negacionestricta}
Una negación estricta es aquella función $c : [0, 1] \rightarrow [0, 1]$ que cumple que $c(0)=1$ y $c(1)=0$ y es estrictamente decreciente y continua. Además, si $c$ es involutiva se considera que se habla de una negación fuerte.
\end{definition}

\begin{definition}\label{def:automorfismo}
Llamaremos automorfismo ($\varphi$) a todas aquellas funciones del intervalo unidad tal que $\varphi : [0, 1] \rightarrow [0, 1]$ sea continua y estrictamente creciente propiciando que $\varphi(0)=0$ y $\varphi(1)=1$.
\end{definition}

En 1979, E. Trillas \cite{art:thtrillas} enunció el siguiente teorema. 
\begin{theorem}[Teorema de Trillas]\label{th:trillas}
Una función $c : [0, 1] \rightarrow [0, 1]$ es una negación fuerte si, y sólo si, existe un automorfismo del intervalo unidad tal que $c(x)=\varphi^{-1}(1-\varphi(x))$.
\end{theorem}

\begin{definition}\label{def:ref}
Una función $REF  : [0, 1] \rightarrow [0, 1]$ es llamada de equivalencia restringida cuando cumple que:
	\begin{enumerate}
	\item $REF(x, y) = REF(y, x), \forall x, y \in [0, 1];$
	\item $REF(x, y) = 1$, si y sólo si, $x=y$;
	\item $REF(x, y) = 0$, si y sólo si, $x=1$ e $y=0$ ó si $x=0$ e $y=1$;
	\item $REF(x, y) = REF(c(x), c(y)),  \forall x, y \in [0, 1]$, siendo $c$ una negación fuerte.
	\item $\forall x, y, z \in [0, 1]$, si $x\leq y\leq z$, entonces $REF(x, y)\geq REF(x, z)$ y  $REF(y, z)\leq REF(x, z)$
	\end{enumerate}
\end{definition}
\begin{proposition}\label{prop:contruccionref}
Para construir funciones $REF$ únicamente necesitaremos de dos automorfismos $\varphi_{1}$ y $\varphi_{2}$ de forma que: 
$$REF(x,y) = \varphi_1^{-1}(1-|\varphi_2(x)-\varphi_2(y)|) \quad\text{con}\quad c(x) = \varphi_2^{-1}(1-\varphi_2(x)).$$ 
Además, si tenemos una REF y un automorfismo en $\unitinterval$, la aplicación de estos ($F=\varphi \circ REF$) es otra REF.
\end{proposition}
\REV{Teorema que hace que estos automorfismos puedan ser 1 solo?}

% FUNCIONES DE AGREGACIÓN
\section{Funciones de agregación}\label{sec:agregacion}
\REV{funciones de agregación vs operadores de agregación}
Las funciones de agregación tienen como propósito reducir las dimensiones de la información a partir de la combinación de los datos de entrada obteniendo una salida que los represente \cite{art:montero, art:calvoagregacion}. Su aplicación se extiende en muchos casos prácticos y teóricos como la lógica multievaluada, control difuso, la toma de decisión, etc \REV{cita}\cite{}.  Se definirá una función de agregación como sigue:

\begin{definition}\label{def:agregacion}
Se dice que $M : \unitinterval^n \rightarrow \unitinterval$ es una función de agregación de dimensión $n$ siempre que satisfaga:
	\begin{enumerate}
	\item $M(x_1, \dots, x_n) = 0$ si y sólo si $x_1=\dots=x_n=0$;
	\item $M(x_1, \dots, x_n) = 1$ si y sólo si $x_1=\dots=x_n=1$;
	\item $M$ es una función estrictamente creciente.
	\end{enumerate}
\end{definition}
\begin{definition}
Una función de agregación $M$ será llamada media si
$$ \min(x_{1}, \dots, x_{n})  \leq M(x_{1}, \dots, x_{n}) \leq \max(x_{1}, \dots, x_{n}).$$
\end{definition}

En este proyecto se utilizarán mayoritariamente las funciones de agregación idempotentes, esto es, que cumplen que $M(x,\dots ,x)=x, \forall x$. Entre las funciones que utilizaremos encontramos la media aritmética, el mínimo, el máximo o la media geométrica. Además, se definen a continuación otras que no son de uso tan común.
% OWA
\subsection{Funciones OWA}
En esta sección se introduce el concepto de las funciones de media ponderada ordenada (OWA, {\em  ordered weighted averaging}) \cite{art:yagerowa, art:paternain, art:bustinceowa}. Basa su idea en generar una media, ordenando primero los elementos a agregar, para luego darles mayor relevancia a una parte de ellos. Este tipo de función generaliza la media aritmética, siendo esta aquella en la que todos los elementos del vector de pesos son iguales.

\begin{definition}\label{def:owa}
Una  función $F:\unitspace{n}$ será una función OWA de dimensión $n$ si existe un vector de pesos $w=(w_{1},w_{2},\dots,w_{n})\in \unitinterval^{n}$ siempre que $\sum_{i}w_{i}=1$ de forma que
$$F(x_{1},\dots,x_{n})=\sum^{n}_{j=1}w_{j}x_{\sigma(j)}$$
sabiendo que $x_{\sigma(j)}$ es el $j$-ésimo mayor elemento del vector $(x_{1},\dots,x_{n})$.
\end{definition}

Por tanto para poder obtener un resultado adecuado, deberá utilizarse un vector de pesos $w$ que se adecue a las necesidades del problema. En este trabajo se emplea la versión `menor que la mitad' que tiene como vector de pesos a $w=(w_{1}\dots w_{i}, w_{i+1}\dots w_{n})$ sabiendo que $\abs{1, \dots, i}=\abs{i+1,\dots,n}$ de forma que $w_{j}=\frac{1}{2n}$ si $1\leq j\leq i$ y $w_{j}=0$ en el resto de los casos.

% CHOQUET
\subsection{Integral Choquet}
Este otro tipo de función \cite{art:choquet, art:sugenochoquet} de agregación pretende dar una nueva de forma de representar un conjunto de datos en una única salida. De esta forma, se define primeramente una medida a través de la cual se calculará la forma en la que cada elemento del conjunto de datos tendrá reelevancia en la agregación final. Para ello definimos el concepto de medida difusa.
\begin{definition}\label{def:medidadifusa}
Dado $U$ un universo finito; $\mathcal{P}(U)$ el conjunto de todos los subcojuntos de $U$. Una medida difusa será aquella función $\mu:\mathcal{P}(U)\rightarrow\unitinterval$ que satisface que:
\begin{enumerate}
	\item $\mu(\emptyset)=0$ y $\mu(U)=1$.
	\item $A\subseteq B \Rightarrow \mu(A)\leq\mu (B), \forall A, B \subseteq U$.
\end{enumerate}
\end{definition}

A continuación definimos la función Integral Choquet conociendo que tomaremos su versión discreta por el contexto en el que se está trabajando.

\begin{definition}\label{def:choquet}
Dada una función $\sigma$  con correspondencia directa con todos los elementos de un vector $x$ de forma que $x_{k}=\sigma(k), \forall k\in\{1,\dots,n\}$. Sabiendo que $x_{\sigma(j)}$ es el $j$-ésimo mayor elemento del vector $(x_{1},\dots,x_{n})$. La integral discreta de Choquet con respecto a la medida difusa $\mu$ es 
$$Ch_{\mu}(x)=\sum_{i=1}^{n}x_{\sigma(i)}(\mu(\{\sigma(i),\dots,\sigma(n)\})-\mu(\{\sigma(i+1),\dots,\sigma(n)\}))$$
tomando la convención de que $\{\sigma(n+1),\sigma(n)\}=\emptyset$.
\end{definition}

\begin{proposition}\label{prop:choque2owa}
Si denotamos a $w_{\sigma, i}^{\mu} = \mu(\{\sigma(i),\dots,\sigma(n)\})-\mu(\{\sigma(i+1),\dots,\sigma(n)\})$ se obtiene la siguiente definición de la integral Choquet en función de los operadores OWA definidos en \ref{def:owa}:
$$\sum_{i=1}^{n} w_{\sigma, i}^{\mu} \cdot x_{\sigma(i)}.$$
\end{proposition}


% FUNCIONES DE SIMILITUD
\section{Funciones de similitud}\label{sec:similitud}
El concepto de similitud \cite{art:fan1, art:fan2} surge cuando queremos medir cómo de parecidos son dos conjuntos difusos. Este concepto es muy parecido al de REF, pero su diferencia radica en que en este caso se utilizan conjuntos y no elementos. Por esta razón, utilizamos las REF como base para definir las funciones de similitud.
\begin{definition}\label{def:similitud}
Dada una función $M$ de agregación (definición \ref{def:agregacion}) y una función $REF$ (definión \ref{def:ref}) llamaremos a $SM$ función de similitud si $SM : \FF(X) \times \FF(X) \rightarrow [0,1]$ está definida tal que $SM(A,B)=M^n_{i=1}REF(\mu_A(x_i), \mu_B(x_i))$
y satisface las siguientes condiciones:
	\begin{enumerate}
	\item $SM(A, B) = SM(B, A), \forall A, B \in \FF(X)$;
	\item $SM(A, A_c) = 0$, si y sólo si A no es difuso;
	\item $SM(A, B) = 1$ si y sólo si A = B;
	\item Si $A\leq B\leq C$, entonces $SM(A, B)\geq SM(A,C)$ y $SM(C, B)\geq SM(C,A)$;
	\item $SM(A_c, B_c) = SM(A,B)$
	\end{enumerate}	
\end{definition}

Durante el desarrollo de este trabajo se buscará la similitud entre conjuntos difusos y el conjunto $\tilda1$. Se define el conjunto $\tilda1 =\{(u_{i}, \mu_{\tilda1}(x)=1)|u_{i}\in U \}$, esto es, aquel en el que todos sus elementos tienen pertenencia absoluta.



% FUNCIONES PENALTI
\section{Funciones penalti}\label{sec:penalti}
Una función penalti  \cite{art:calvo} es una función de agregación, por lo que dispone de un vector de entradas del que devuelve un único resultado. Si todos los elementos del vector son iguales, entonces, claramente, la salida será eso mismo. Ahora bien, el problema surge cuando hay algún elemento diferente ya que en ese momento la idea será buscar una salida todo lo parecida posible a la entrada. En este sentido, es elemento o elementos diferentes tendrán una cierta discrepancia con los demás y justamente esto es lo que se pretende minimizar, la discrepancia, para dar una salida adecuada. 

\begin{definition}\label{def:penalti}
La función $P:[0,1]^{n+1}\REV{?}\rightarrow \RR^{+} = [+, \infty]$ es una función penalti si y sólo si satiface que:
\begin{enumerate}
	\item $P(x, y) \geq 0, \forall x, y$
	\item $P(x, y) = 0$ si $x_{i}=y \forall i=1,\dots ,n$
	\item para cualquier $x$ fijado, el conjunto de minimizadores de $P(x, y)$ o bien un \REV{producto único} o un intervalo. \REV{reescrito} $P(x,y)$ es cuasiconvexa en $y$ para cualquier $x$, esto es, $P(x, \lambda\cdot y_{1} +(1-\lambda)\cdot y_{2})\leq \max(P(x, y_{1}), P(x, y_{2}))$.
\end{enumerate}
\end{definition}
La función en la que se basan las penalti es $$f(x)=\arg\min_{y} P(x,y)$$ si $y$ es el único mínimo e $y=\frac{a+b}{2}$ si el conjunto de minimizadores es el intervalo $(a, b)$.

\begin{theorem}
Todas las funciones de agregación llamadas medias pueden ser escritas como una función basada en una función penalti expresada en la definición \ref{def:penalti}.
\end{theorem}


% FUNCIONES DE DOMBI
\section{Funciones de Dombi}\label{sec:dombi}
Las funciones de Dombi \cite{art:dombi} son operadores de equivalencia con una definición diferente a las REF presentadas en la sección \ref{sec:ref}. Con esta relación lo que se pretende conocer es como de iguales son dos elementos de un conjunto difuso dado, pero aplicando nuevas ideas para la construcción del resultado. 
\REV{Definición completa?}
\begin{definition}\label{def:dombi}
Denotaremos $D$ como una función de equivalencia de Dombi cuando tengamos que 
$$D(w,x)=\frac{1}{2}\left(1+\prod(1-2x_{i})^{w_{i}}\right)$$
\end{definition}
\REV{propiedades}

\REV{Existe también el operador con pesos}



%NOTACIÓN
\section{Notación}\label{sec:notacion}

A lo largo del trabajo se asumirá la notación que sigue para estos conceptos.

\begin{description}
    \item[Imagen]: la denotaremos con $Q$.
    \item[Coordenadas de un pixel]: $(x,y)$.
    \item[Máximo nivel de gris]: $L$.
    \item[Intensidad de un pixel]: $q(x,y)$ de forma que $0\leq q(x,y)\leq L-1, \forall (x,y)\in Q$.  
    \item[Histograma]: $h(q)$. Función para conocer el número de pixeles con la intensidad $q$. 
    \item[Media de una imagen]:$$m_Q=\frac{\sum_{q=0}^{L-1}qh(q)}{\sum_{q=0}^{L-1}h(q)}$$
    \item[Área de una imagen]: $$A(Q) = \sum_{q=0}^{L-1} q h(q)$$
\end{description}























