\documentclass[main]{subfiles}

\begin{document}

% FUNCIONES REF
\section{Funciones de equivalencia restringida (REF)}\label{sec:ref}
El concepto de función de equivalencia restringida (REF por sus siglas en inglés) surge del concepto de equivalencia y el de similitud \cite{art:refbarrenechea}. Este concepto es muy utilizado para la comparación de imágenes y su intención es dar una medida de cómo de iguales o similares son dos elementos $x$ e $y$. Para poder definir el concepto de REF, necesitamos varios previamente.

\begin{definition}\label{def:negacionestricta}
Una negación estricta es aquella función $c : [0, 1] \rightarrow [0, 1]$ que cumple que $c(0)=1$ y $c(1)=0$ y es estrictamente decreciente y continua. Además, si $c$ es involutiva se considera que se habla de una negación fuerte.
\end{definition}

\begin{definition}\label{def:automorfismo}
Llamaremos automorfismo ($\varphi$) a todas aquellas funciones del intervalo unidad tal que $\varphi : [0, 1] \rightarrow [0, 1]$ sean continuas y estrictamente crecientes propiciando que $\varphi(0)=0$ y $\varphi(1)=1$.
\end{definition}

En 1979, E. Trillas \cite{art:thtrillas} enunció el siguiente teorema.
\begin{theorem}[Teorema de Trillas]\label{th:trillas}
Una función $c : [0, 1] \rightarrow [0, 1]$ es una negación fuerte si, y sólo si, existe un automorfismo del intervalo unidad tal que $c(x)=\varphi^{-1}(1-\varphi(x))$.
\end{theorem}

\begin{definition}\label{def:ref}
Una función $REF  : [0, 1] \rightarrow [0, 1]$ es llamada de equivalencia restringida cuando cumple que:
	\begin{enumerate}
	\item $REF(x, y) = REF(y, x), \forall x, y \in [0, 1];$
	\item $REF(x, y) = 1$, si y sólo si, $x=y$;
	\item $REF(x, y) = 0$, si y sólo si, $x=1$ e $y=0$ ó si $x=0$ e $y=1$;
	\item $REF(x, y) = REF(c(x), c(y)),  \forall x, y \in [0, 1]$, siendo $c$ una negación fuerte.
	\item $\forall x, y, z \in [0, 1]$, si $x\leq y\leq z$, entonces $REF(x, y)\geq REF(x, z)$ y  $REF(y, z)\leq REF(x, z)$
	\end{enumerate}
\end{definition}

\begin{proposition}\label{prop:contruccionref}
Sean dos automorfismos $\varphi_{1}$ y $\varphi_{2}$, se llamará función $REF$ a la construcción que cumpla que:
$$REF(x,y) = \varphi_1^{-1}(1-|\varphi_2(x)-\varphi_2(y)|) \quad\text{con}\quad c(x) = \varphi_2^{-1}(1-\varphi_2(x)).$$
Además, si tenemos una $REF$ y un automorfismo en $\unitinterval$, la aplicación de estos ($F=\varphi \circ REF$) es otra $REF$.
\end{proposition}

\begin{theorem}\label{th:ref}
Una función continua $REF:\unitinterval^2 \rightarrow\unitinterval$ que cumpla que $REF(1,x)=x, \xinunitinterval$, es una función $REF$ asociada con la función $I:\unitinterval^2\rightarrow\unitinterval$ con $I$ continua y asociativa si y sólo si existe un automorfismo $\varphi$ en $\unitinterval$ tal que:
$$REF(x,y)=\varphi^{-1}(1-\abs{(\varphi(x)-\varphi(y))})\quad\text{ y }\quad c(x)=\varphi^{-1}(1-\varphi(x))$$
\end{theorem}

%\REV{Teorema que hace que estos automorfismos puedan ser 1 solo?} No lo pongo pq no lo utilizo.

\end{document}
