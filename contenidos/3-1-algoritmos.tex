
% ALGORITMO 1 GENERAL
\subsection{Algoritmo general maximizando la similitud}

Este algoritmo, que se presenta en \cite{art:barrenechea}, pretende conseguir obtener un único umbral a partir de la maximización de la similitud. 

\begin{algorithm}
\begin{algorithmic}[1]
\REQUIRE Una imagen $Q$ en escala de grises donde sus píxeles estén entre $0$ y $L-1$.
\ENSURE El umbral $t$ a partir del cual se divide $Q$ en objeto y fondo.
\FOR {$t$:=0 hasta $L-1$}
\STATE Divisón de la imagen en dos clases $C_b(t)$ y $C_o(t)$. Para cada una de estas clases, calcular su media: $m_b(t)$ y $m_o(t)$.
\STATE Construcción del conjunto difuso $Q_t$.
\STATE Calcular la $SM(\tilda1, Q_t)$. \label{lin:alg1:similitud}
\ENDFOR
\RETURN \{$t$ | max($SM$)\}
\end{algorithmic}
\caption{Maximización de la similitud}\label{alg:algoritmo1}
\end{algorithm}

En el algoritmo anterior se necesitan varias definiciones que se explican a continuación. En primer lugar, se describe el método de creación de los conjuntos difusos para lo que se explica previamente el cálculo de las medias para el fondo y el objeto.

\begin{definition}\label{def:mediasmonoumbral}
Teniendo en cuenta la definición de la media de una imagen que se ha dado en el apartado \ref{sec:notacion}, y disponiendo del histograma de la imagen $h(q)$ para un cierto nivel $q, \forall q\in Q$, se define la media de los píxeles del fondo como:
$$m_b(t)=\frac{\sum_{q=0}^{t}qh(q)}{\sum_{q=0}^{t}h(q)};$$
y para los píxeles del objeto como:
$$m_o(t)=\frac{\sum_{q=t+1}^{L-1}qh(q)}{\sum_{q=t+1}^{L-1}h(q)}.$$
\end{definition}

\begin{definition}\label{def:conjuntodifusomonoumbral}
Dada $Q$, una imagen en la escala de L niveles de gris, y $t$, un nivel de gris de forma que $0\leq t\leq L-1$. Teniendo en cuenta que $F$ es una función $REF$ ya que la $REF \circ \varphi$ lo es, se define el conjunto
$$Q_t = \{(q, \mu_{Q_t}(q)|q\in \{0,1,\dots, L-1\}\}$$ 
teniendo en cuenta que
$$\mu_{Q_t}(q) = \left\{ \begin{aligned}
    F \left(\frac{q}{L-1}, \frac{m_b(t)}{L-1} \right) = \varphi\left(REF\left(\frac{q}{L-1}, \frac{m_b(t)}{L-1} \right)\right) & \quad\text{si}\ q\leq t,\\
    F \left(\frac{q}{L-1}, \frac{m_o(t)}{L-1} \right) = \varphi\left(REF\left(\frac{q}{L-1}, \frac{m_o(t)}{L-1} \right)\right) & \quad\text{si}\ q> t.
 \end{aligned}\right.$$
 \end{definition}

Como parece lógico a la vista de la definición \ref{def:conjuntodifusomonoumbral}, es necesario poder enumerar aquellas funciones que se utilizarán como $REF$ y como automorfismo $\varphi$. Para todos los casos se tendrá $\varphi = x$ y, además, se distinguirán las siguientes funciones $REF$:
\begin{enumerate}
    \item $REF(x,y)=1-\abs{x-y}$
    \item $REF(x,y)=1-\abs{x-y}^2$
    \item $REF(x,y)=1-\abs{x-y}^{0.5}$
    \item $REF(x,y)=(1-\abs{x-y})^2$
    \item $REF(x,y)=(1-\abs{x-y})^{0.5}$
\end{enumerate}

Debido a la formulación de este tarbajo, se sustituirá la aplicación anterior que actúa como función $F$ en la definición \ref{def:conjuntodifusomonoumbral} por la función de Dombi \cite{art:dombi} definidas en \ref{def:dombi}. De esta forma, en este caso, los conjuntos difusos quedarán como explica la ecuación \ref{eq:conjdifusosdombimono}.

\begin{equation} \label{eq:conjdifusosdombimono}
    \mu_{Q_t}(q) = \left\{ \begin{aligned}
        F \left(w,\left(\frac{q}{L-1}, \frac{m_b(t)}{L-1}\right)\right) = \frac{1}{2}\left(1 + \left(1-2\frac{q}{L-1}\right)^w\cdot\left(1-2\frac{m_b(t)}{L-1}\right)^w\right)& \quad\text{si}\ q\leq t,\\
        F \left(w,\left(\frac{q}{L-1}, \frac{m_o(t)}{L-1}\right)\right) = \frac{1}{2}\left(1 + \left(1-2\frac{q}{L-1}\right)^w\cdot\left(1-2\frac{m_o(t)}{L-1}\right)^w\right)& \quad\text{si}\ q> t.
     \end{aligned}\right.
\end{equation}
\REV{Está bien definido cómo he metido el parámetro d??}
Además, el último paso del bucle, en la línea \ref{lin:alg1:similitud}, se lleva a cabo la busqueda de la similitud frente al conjunto $\tilda1$. Para poder llevar a cabo este cálculo, tal y como se especifica en la definición \ref{def:similitud}, necesitamos utilizar una función $REF$, que será $1-\abs{x-y}^2$ y la agregación $M$, la media aritmética ($\frac{1}{n}\sum_1^n x_i$). De esta forma, quedará como sigue:
\begin{equation}\label{eq:similitud}
    SM(\tilda1, Q_{t}) = M^{L-1}_{q=0}(h(q)REF_2(1,\mu_{Q_t}(q)))
\end{equation}


% ALGORITMO DEL ÁREA
\subsection{Algoritmo del área}

Este algoritmo, que se presenta en \cite{art:barrenechea}, pretende hayar un nuevo umbral a través de la creación de una función $REF$ 

\begin{algorithm}
\begin{algorithmic}[1]
\REQUIRE Una imagen $Q$ en escala de grises donde sus píxeles estén entre $0$ y $L-1$.
\ENSURE El umbral $t$ a partir del cual se divide $Q$ en objeto y fondo.
\FOR {$t$:=0 hasta $L-1$}
\STATE \begin{equation*}\begin{split}
A(Q_t)= \sum_{q=0}^{t} h(q)\varphi_1^{-1}\left(1-\abs{\varphi_2\left(\frac{q}{L-1}\right)-\varphi_2\left(\frac{m_b(t)}{L-1}\right)}\right) + \\ \sum_{q=t+1}^{L-1} h(q)\varphi_1^{-1}\left(1-\abs{\varphi_2\left(\frac{q}{L-1}\right)-\varphi_2\left(\frac{m_o(t)}{L-1}\right)}\right)
\end{split}\end{equation*}
\ENDFOR
\RETURN \{$t$ | max($A$)\}
\end{algorithmic}
\caption{Umbralización del área}\label{alg:algoritmo2}
\end{algorithm}

En este algoritmo, utilizaremos 4 versiones, en función del par de automorfismos que utilicemos. 

%\begin{table}[h!]\begin{center}
%\resizebox*{14cm}{!}{\begin{tabular}{c||c|c||c} 
%&$\varphi_1$ & $\varphi_2$ & $A(Q_t)$\\\hline\hline
%(1)&$x$ & $x$ & $\sum_{q=0}^{L-1} h(q) - \left(\sum_{q=0}^{t} \left(1-\abs{\frac{q}{L-1}-\frac{m_b(t)}{L-1}}\right) - \sum_{q=t+1}^{L-1} \left(1-\abs{\frac{q}{L-1}-\frac{m_o(t)}{L-1}}\right)\right)$ \\\hline
%(2)&$x^d$ & $x$ & $\sum_{q=0}^{t} h(q) \left(1-\abs{\frac{q}{L-1}-\frac{m_b(t)}{L-1}}\right) - \sum_{q=t+1}^{L-1} h(q) \left(1-\abs{(\frac{q}{L-1}-\frac{m_o(t)}{L-1}}\right)$ \\\hline

%(3)&$x^1-(1-x)^{1/2}$ & $x$ & $\sum_{q=0}^{L-1} h(q) - \left(\sum_{q=0}^{L-1} h(q) \left(\frac{q}{L-1}-\frac{m_b(t)}{L-1}\right)^2 - \sum_{q=t+1}^{L-1} h(q) \left(\frac{q}{L-1}-\frac{m_o(t)}{L-1}\right)^2\right)$ \\\hline

%\end{tabular}}\end{center}
%\caption{Porcentajes de acierto para los diferentes \datasets y configuraciones para {\em train}.\label{resultTrain}}
%\end{table}

\begin{enumerate}\label{enum:funcionesalg2}
    \item Tomando $\varphi_1(x) = \varphi_2(x) = x, \xinunitinterval$.
    $$A(Q_t) = sum_{q=0}^{L-1} h(q) - \left(\sum_{q=0}^{t} \left(1-\abs{\frac{q}{L-1}-\frac{m_b(t)}{L-1}}\right) - \sum_{q=t+1}^{L-1} \left(1-\abs{\frac{q}{L-1}-\frac{m_o(t)}{L-1}}\right)\right)$$
    \item Tomando $\varphi_1(x) = x^d \text{ con } d\neq0, \xinunitinterval \text{ y } \varphi_2(x)=x, \xinunitinterval$.
    $$\sum_{q=0}^{t} h(q) \left(1-\abs{\frac{q}{L-1}-\frac{m_b(t)}{L-1}}\right) - \sum_{q=t+1}^{L-1} h(q) \left(1-\abs{(\frac{q}{L-1}-\frac{m_o(t)}{L-1}}\right)$$
    \item Tomando $\varphi_1(x) = 1-\sqrt{1-x}, \xinunitinterval \text{ y } \varphi_2(x)=x,\xinunitinterval$
    $$\sum_{q=0}^{L-1} h(q) - \left(\sum_{q=0}^{L-1} h(q) \left(\frac{q}{L-1}-\frac{m_b(t)}{L-1}\right)^2 - \sum_{q=t+1}^{L-1} h(q) \left(\frac{q}{L-1}-\frac{m_o(t)}{L-1}\right)^2\right)$$
\end{enumerate}

% ALGORITMO 3
\subsection{Algoritmo de selección del umbral óptimo}

\begin{algorithm}
\begin{algorithmic}[1]
\REQUIRE Una imagen $Q$ en escala de grises donde sus píxeles estén entre $0$ y $L-1$.
\ENSURE El umbral óptimo $t*$ a partir del cual se divide $Q$ en objeto y fondo.
\FOR {$t$:=0 hasta $L-1$}
\STATE Calcular los conjuntos $Q_t$ como se describe en la definición \ref{def:conjuntodifusomonoumbral}.
\STATE Calcular los conjuntos $H$ como se muestra en la definición \ref{def:conjuntoHmonoumbral}.
\STATE Calcular la $SM(Q_t, H(Q_t))$.
\ENDFOR
\RETURN \{$t*$ | max($SM$)\}
\end{algorithmic}
\caption{Selección del umbral optimo}\label{alg:algoritmo3}
\end{algorithm}

\begin{definition}\label{def:conjuntoHmonoumbral}
Dada $Q$, una imagen en la escala de L niveles de gris, y $t$, un nivel de gris de forma que $0\leq t\leq L-1$, se calcula su conjunto $H(Q_t)$ como
$$H(Q_t) = \{(q, \mu_{H(Q_t)}(q)|q\in \{0,1,\dots, L-1\}\}$$ 
teniendo en cuenta que
$$\mu_{Q_t}(q) = \left\{ \begin{aligned}
    \frac{m_b(t)}{L-1} & \quad\text{si}\ q\leq t,\\
    \frac{m_o(t)}{L-1} & \quad\text{si}\ q> t.
 \end{aligned}\right.$$
 \end{definition}

Además:
$$SM(Q_t, H(Q_t)) = M^{L-1}_{q=0}(h(q)REF(\mu_{Q_t}(q), \mu_{H(Q_t)}(q)))$$


%ALGORITMO GLOBAL
\subsection{Otros algoritmos}
\subsubsection{Algoritmo de umbralización global}
\begin{algorithm}
\begin{algorithmic}[1]
\REQUIRE Una imagen $Q$ en escala de grises donde sus píxeles estén entre $0$ y $L-1$.
\ENSURE El umbral óptimo $t$ a partir del cual se divide $Q$ en objeto y fondo.
\STATE t0 = 128; 
\COMMENT{Se selecciona un umbral inicial cualquiera, aquí se tomará el valor medio de los posibles}
\REPEAT
\STATE $G1 = \{(x, y) | q(x, y) > t0\}$
\STATE $G2 = \{(x, y) | q(x, y) \leq t0\}$
\STATE $m1 = \frac{1}{\abs{G1}} \sum_{i\in G1} i$
\STATE $m2 = \frac{1}{\abs{G2}} \sum_{i\in G2} i$
\STATE $t = \frac{m1+m2}{2}$
\UNTIL {$\abs{t0-t} < \varepsilon$}
\COMMENT {El valor $\varepsilon$ es un cierto error dispuesto por el programador.}
\RETURN $t$
\end{algorithmic}
\caption{Umbralización global.}\label{alg:global}
\end{algorithm}


% ALGORTIMO DE OTSU
\subsubsection{Algoritmo de Otsu}

El algoritmo de Otsu \cite{art:otsu}...

\begin{algorithm}
\begin{algorithmic}[1]
\REQUIRE Una imagen $Q$ en escala de grises donde sus píxeles estén entre $0$ y $L-1$.
\ENSURE El umbral óptimo $t*$ a partir del cual se divide $Q$ en objeto y fondo.
\FOR {$t$:=0 hasta $L-1$}
\STATE $\omega_0$(t) = $\sum_{i=1}^{t}p_i$
\STATE $\mu_0$(t) = $\sum_{i=1}^t \frac{ip_i}{\omega_0}$
\STATE $\omega_1$(t) = $\sum_{i=t+1}^{L}p_i$
\STATE $\mu_1$(t) = $\sum_{i=t+1}^L \frac{ip_i}{\omega_1}$
\STATE $\sigma_B^2(t) = \omega_0\omega_1(\mu_1-\mu_0)^2$
\ENDFOR
\RETURN \{$t*$ | max_t($\sigma_B^2$)\}
\end{algorithmic}
\caption{Selección del umbral óptimo de Otsu.}\label{alg:otsu}
\end{algorithm}

$$p_i = \frac{n_i}{N}, \quad p_i \geq 0, \sum_{i=1}^{L}p_i=1$$
$$\omega_0 = \Pr(C_0) = \sum_{i=1}^{t}p_i=\omega_0(t)$$
$$\omega_1 = \Pr(C_1) = \sum_{i=t+1}^{L}p_i=\omega_1(t)$$
$$\mu_0=\sum_{i=1}^t i \Pr(i|C_0) = \sum_{i=1}^t \frac{ip_i}{\omega_0} = \frac{\mu(t)}{\omega(t)}$$
$$\mu_1=\sum_{i=t+1}^{L} i \Pr(i|C_1) = \sum_{i=t+1}^{L} \frac{ip_i}{\omega_1} = \frac{\mu_T-\mu(t)}{1-\omega(t)}$$
$$\mu(k)=\sum_{i=1}^k ip_i$$
$$\mu_T = \mu(L) = \sum_{i=1}^L ip_i$$
$$\omega_0\mu_0+\omega_1\mu_1 = \mu_T$$
$$\omega_0+\omega_1 = 1$$
$$\sigma_B^2(t) = \omega_0(\mu_0-\mu_T)^2 + \omega_1(\mu_1-\mu_T)^2 = \omega_0\omega_1(\mu_1-\mu_0)^2$$
$$\sigma_B^2(t*) = \max_{1\leq k <L}\sigma_B^2(t)$$
\REV{El umbral de 1..L en vez de 0..L-1}

% ALGORITMO DE LA ENTROPÍA DE RENYI
\subsubsection{Algoritmo de maximización de la entropía de Renyi}
Este algoritmo enunciado en \cite{art:sahoo}...



Método de el anterior KAPUT
$$H_T=-\sum_{i=0}^{L-1} p_i \ln p_i$$
$$H_{C_0}(t)=-\sum_{i=0}^{t} \frac{p_i}{p(C_0)} \ln \frac{p_i}{p(C_0)}$$
$$H_{C_1}(t)=-\sum_{i=t+1}^{L-1} \frac{p_i}{p(C_1)} \ln \frac{p_i}{p(C_1)}$$
$$p(C_0)\sum_{i=0}^{t}p_i$$
$$p(C_1)\sum_{i=0}^{t}p_i$$
$$p(C_0)+p(C_1)=1$$

Método correcto
$$C_0:\frac{p_0}{p(C_0)},\frac{p_1}{p(C_0)},\dots,\frac{p_t}{p(C_0)}$$
$$C_1:\frac{p_0}{p(C_1)},\frac{p_1}{p(C_1)},\dots,\frac{p_t}{p(C_1)}$$
$$H_{T}^{\alpha}(t) = \frac{1}{1-\alpha}\ln \sum_{i=0}^{L-1}\left(p_i\right)^\alpha \quad\text{con }\alpha\neq 1$$
$$H_{C_0}^{\alpha}(t) = \frac{1}{1-\alpha}\ln \sum_{i=0}^{t}\left(\frac{p_i}{p(C_0)}\right)^\alpha$$
$$H_{C_1}^{\alpha}(t) = \frac{1}{1-\alpha}\ln \sum_{i=t}^{L-1}\left(\frac{p_i}{p(C_1)}\right)^\alpha$$
$$t^*(\alpha)=\max_{0\leq t<L}\left\{H_{C_0}^{\alpha}(t)+H_{C_1}^{\alpha}(t)\right\}$$
$$t^*(\alpha)=\left\{\begin{aligned}
    t^*_1 & \text{  si  }0<\alpha<1,\\
    t^*_2 & \text{  si  }\alpha\rightarrow1,\\
    t^*_3 & \text{  si  }1<\alpha<\infty.
\end{aligned}\right.$$
$$t^*_c = t_{(1)} \left(p(t_{(1)})+\frac{1}{4}\omega\beta_1\right)
        + \frac{1}{4}t_{(2)}\omega\beta_2
        + t_{(3)} \left(1-p(t_{(3)})+\frac{1}{4}\omega\beta_3\right)$$
donde...
$$p(t)=\sum_{i=1}^t(p_i); \omega=p(t_{(3)})-p(t_{(1)})$$
y...
$$(\beta_1, \beta_2, \beta_3) =  \left\{\begin{aligned}
    (1, 2, 1) & \quad\text{si} &\abs{\tonee-\ttwo}\leq 5 \text{ y } \abs{\ttwo-\tthree}\leq 5,\\
    (1, 2, 1) & \quad\text{si} &\abs{\tonee-\ttwo}  >  5 \text{ y } \abs{\ttwo-\tthree}  >  5,\\
    (0, 1, 3) & \quad\text{si} &\abs{\tonee-\ttwo}\leq 5 \text{ y } \abs{\ttwo-\tthree}  >  5,\\
    (3, 1, 0) & \quad\text{si} &\abs{\tonee-\ttwo}  >  5 \text{ y } \abs{\ttwo-\tthree}\leq 5,   
\end{aligned}\right.$$

por ello, se sabe que:
$$\min\{t_1^*, t^*_2, t^*_3 \}\leq t_c^* \leq\max\{t_1^*, t^*_2, t^*_3\} = \tonee \leq t_c^* \leq \tthree$$

\begin{algorithm}
\begin{algorithmic}[1]
\REQUIRE Una imagen $Q$ en escala de grises donde sus píxeles estén entre $0$ y $L-1$.
\ENSURE El umbral óptimo $t*$ a partir del cual se divide $Q$ en objeto y fondo.

\STATE $\alpha = [0.3, 0.99999, 10]$
\COMMENT{Para facilitar la implementación, tendremos en cuenta únicamente un alfa de cada uno de los casos, cuando se estabilizan cada uno de los casos}
\FOR {$\alpha_i \in \alpha$}
    \STATE $H_T=0$
\FOR {$t'=0$ hasta $L-1$}
        \STATE $p_{C_0} = \sum_{i=0}^{t'}p_i$
        \STATE $p_{C_1} = \sum_{i=t'+1}^{L-1}p_i$
        \STATE $H_{C_0}^{\alpha}(t') = \frac{1}{1-\alpha}\ln \sum_{i=0}^{t'}\left(\frac{p_i}{p(C_0)}\right)^\alpha$
        \STATE $H_{C_1}^{\alpha}(t') = \frac{1}{1-\alpha}\ln \sum_{i=t'}^{L-1}\left(\frac{p_i}{p(C_1)}\right)^\alpha$
        \STATE $H_T(t') = H_T(t') + H_{C_0} +H_{C_1}$
     \ENDFOR
    \STATE $t_{mejor}(i) = \max_{0\leq j<L}\left\{H_T(j)\right\}$
\ENDFOR
\STATE $t$ = ordenar($t_{mejor}$);
\STATE $\omega =\sum_{i=0}^{\tthree}p_i - \sum_{i=0}^{\tonee}p_i$
\IF {$\abs{\tonee-\ttwo}\leq 5$ y $\abs{\ttwo-\tthree}\leq 5$}
    \STATE $\beta$ = [1, 2, 1];
\ELSIF {$\abs{\tonee-\ttwo} > 5$ y $\abs{\ttwo-\tthree} > 5$}
    \STATE $\beta$ = [1, 2, 1];
\ELSIF {$\abs{\tonee-\ttwo}\leq 5$ y $\abs{\ttwo-\tthree} > 5$}
    \STATE $\beta$ = [0, 1, 3];
\ELSIF {$\abs{\tonee-\ttwo} > 5$ y $\abs{\ttwo-\tthree}\leq 5$}
    \STATE $\beta$ = [3, 1, 0];
\ENDIF
\STATE $t^*_c = t_{(1)} \left(p(t_{(1)})+\frac{1}{4}\omega\beta_1\right) + \frac{1}{4}t_{(2)}\omega\beta_2 + t_{(3)} \left(1-p(t_{(3)})+\frac{1}{4}\omega\beta_3\right)$
\RETURN $t^*$
\end{algorithmic}
\caption{Selección del umbral óptimo maximizando la entropía de Renyi.}\label{alg:renyi}
\end{algorithm}


% ALGORTIMO K-MEANS
\subsubsection{Algoritmo {\em k-means}}


\begin{algorithm}
\begin{algorithmic}[1]
\REQUIRE Una imagen $Q$ en escala de grises donde sus píxeles estén entre $0$ y $L-1$ y el número de clases $K$ que se desean obtener.
\ENSURE La imagen umbralizada en dos regiones con dos tonos de gris diferenes, $imgSegmentada$.
\STATE N = filas(Q);
\STATE M = columnas(Q);
\STATE $\mu$ = aleatorios(K);
\STATE J = 0;
\COMMENT {Coste}
\REPEAT 
    \FOR {$i=1$ hasta N}
        \FOR {$j=1$ hasta M}
            \STATE Jaux(i,j) = $\min_{j=1,\dots,K} \abs{\abs{q(i,j)-\mu(j)}}^2;$
        \ENDFOR
    \ENDFOR
    \FOR {$j=1$ hasta K}
        \STATE $\mu(j) = \frac{1}{\abs{cluster_j}}\sum_{i\in cluster_j}q_i$  \REV{esto debería ser de i, j la pertenencia al cluster}
    \ENDFOR 
\UNTIL {J $\neq$ Jant}

\FOR {$j:=1$ hasta $K$}
    \STATE imgSegmentada = imgSegmentada + $\sum_{i\in cluster_j}\mu(i)$
\ENDFOR
\RETURN imgSegmentada;
\end{algorithmic}
\caption{Umbralización por medio de {\em $k$-means}.}\label{alg:kmeans}
\end{algorithm}


%coste
%$$J = \sum_{i=0}^N \sum_{j=0^M} \abs{\abs{q(i,j) - \mu_{c_i}}}^2$$
%
%$$c_{i,j} =\min_{j=1,\dots,K} \abs{\abs{q(i,j)-\mu(j)}}^2$$
% 
%$$\frac{d}{d \mu_j} = \sum_{i\in cluster_j} \abs{\abs{q(i,j)-\mu(j)}}^2 = -2\sum_{i\in cluster_j}(q(i,j)-\mu(j))=0$$
%$$\mu(j) = \frac{1}{\abs{cluster_j}} \sum_{i\in cluster_j}q_i$$





Hemos quedado en que hay que calcular t+1 pesos. Para ello se dan las funciones que siguen:

\begin{equation}
Q(r) = \left\{\begin{aligned}
    0 & \quad\text{si} \quad& r<0,5\\
    \frac{r-0,5}{0,5} & \quad\text{si} & 0,5\leq r \leq 1\\
    1 & \quad\text{si}\quad & r > 1
\end{aligned}\right.
\end{equation}
y construimos los pesos...
$$w_i=Q\left(\frac{i}{t+1}\right) - Q\left(\frac{i-1}{t+1}\right)
\text{ donde }\sum w_i=1$$

