\documentclass[main]{subfiles}

\begin{document}

%NOTACIÓN
\section{Notación}\label{sec:notacion}

A lo largo del trabajo se asumirá la notación que sigue para estos conceptos.

\begin{description}
    \item[Imagen]: la denotaremos con $Q$.
    \item[Coordenadas de un pixel]: $(x,y)$.
    \item[Máximo nivel de gris]: $L$.
    \item[Número de filas de $Q$]: $N$.
    \item[Número de columnas de $Q$]: $M$.
    \item[Intensidad de un pixel]: $q(x,y)$ de forma que $0\leq q(x,y)\leq L-1, \forall (x,y)\in Q$.
    \item[Histograma]: $h(q)$. Función para conocer el número de pixeles con la intensidad $q$.
    \item[Media de una imagen]:$$m_Q=\frac{\sum_{q=0}^{L-1}qh(q)}{\sum_{q=0}^{L-1}h(q)}$$
    \item[Área de una imagen]: $$A(Q) = \sum_{q=0}^{L-1} q h(q)$$
\end{description}

\end{document}
