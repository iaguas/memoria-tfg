\documentclass[main]{subfiles}

\begin{document}

% FUNCIONES DE AGREGACIÓN
\section{Funciones de agregación}\label{sec:agregacion}
%\REV{funciones de agregación vs operadores de agregación ¿Qué es más correco para nombrarlos?} -> Funciones de agregación.
Las funciones de agregación tienen como propósito reducir las dimensiones de la información a partir de la combinación de los datos de entrada obteniendo una salida que los represente \cite{art:montero, art:calvoagregacion}. Su aplicación se extiende en muchos casos prácticos y teóricos como la lógica multievaluada, control difuso, la toma de decisión, etc \cite{art:paternain}.  Se definirá una función de agregación como sigue:

\begin{definition}\label{def:agregacion}
Se dice que ${M : \unitinterval^n \rightarrow \unitinterval}$ es una función de agregación de dimensión $n$ siempre que satisfaga:
	\begin{enumerate}
	\item $M(x_1, \dots, x_n) = 0$ si y sólo si $x_1=\dots=x_n=0$;
	\item $M(x_1, \dots, x_n) = 1$ si y sólo si $x_1=\dots=x_n=1$;
	\item $M$ es una función estrictamente creciente.
	\end{enumerate}
\end{definition}
\begin{definition}
Una función de agregación $M$ será llamada media si
$$ \min(x_{1}, \dots, x_{n})  \leq M(x_{1}, \dots, x_{n}) \leq \max(x_{1}, \dots, x_{n}).$$
\end{definition}

En este proyecto se utilizarán mayoritariamente las funciones de agregación idempotentes, esto es, que cumplen que $M(x,\dots ,x)=x, \forall x$. En particular, una función de agregación idempotente es una media. Entre las funciones que utilizaremos encontramos la media aritmética, el mínimo, el máximo o la media geométrica. Además, se definen a continuación otras que no son de uso tan común.
% OWA
\subsection{Funciones OWA}
En esta sección se introduce el concepto de las funciones de media ponderada ordenada (OWA, {\em  ordered weighted averaging}) \cite{art:yagerowa, art:paternain, art:bustinceowa}. Se basan en la idea de generar una media, ordenando primero los elementos a agregar, para luego darles mayor relevancia a una parte de ellos. Este tipo de función generaliza la media aritmética, siendo esta el OWA en el que todos los elementos del vector de pesos son iguales.

\begin{definition}\label{def:owa}
Una  función $F:\unitinterval^n\rightarrow\unitinterval$ será una función OWA de dimensión $n$ si existe un vector ${w=(w_{1},w_{2},\dots,w_{n})\in \unitinterval^{n}}$ tal que ${\sum_{i}w_{i}=1}$ de forma que
$$F(x_{1},\dots,x_{n})=\sum^{n}_{j=1}w_{j}x_{\sigma(j)}$$
donde ${x_{\sigma(j)}}$ es el $j$-ésimo mayor elemento del vector ${(x_{1},\dots,x_{n})}$.
\end{definition}

Por tanto para poder obtener un resultado adecuado, deberá utilizarse un vector de pesos $w$ que se adecúe a las necesidades del problema. En este trabajo se emplea la versión `al menos la mitad' %\REV{nombre}. Solucionado.
 que tiene como vector de pesos aquel que se obtiene con la ecuación \ref{eq:pesosowamayoria}. Esto generará una agregación donde destaquen todos aquellos elementos que se encuentren por detras de la mediana.
% a $w=(w_{1}\dots w_{i}, w_{i+1}\dots w_{n})$ sabiendo que $\abs{1, \dots, i}=\abs{i+1,\dots,n}$ de forma que $w_{j}=\frac{1}{2n}$ si $1\leq j\leq i$ y $w_{j}=0$ en el resto de los casos.\REV{buscar la versión buena, esto es sólo para los cardinales pares}.\REV{volver a redactar. Incluir construcción.}

\begin{equation}\label{eq:pesosowamayoria}
	 w_i = Q\left(\frac{i}{t+1}\right) - Q\left(\frac{i+1}{t+1}\right), \forall i\in \{1, \dots, n\}, \text{   sabiendo que}
\end{equation}
	 $$Q(r) = \left\{\begin{aligned}
	 	0 					&\quad \text{si}\quad r<0,5\\
	 	\frac{r-0,5}{0,5}	&\quad \text{si}\quad 0,5\leq r\leq 1\\
	 	1 					&\quad \text{si}\quad r>1\\
	\end{aligned}\right.$$
%\REV{forma ecuación}

%\REV{reescrito}
Además, en la segunda parte del estudio, se emplea el OWA `la mayoría de'. Este OWA pretende destacar aquellos elementos que son mayores en el vector $x$. Esto se hace realidad por medio de un vector de pesos donde son nulos los pesos relativos a aquellas $x_i$ más pequeñas. Para la construcción del vector de pesos, se utilizará, también, la ecuación \ref{eq:pesosowamayoria}



% CHOQUET
\subsection{Integral Choquet}
Para este otro tipo de función \cite{art:choquet, art:sugenochoquet} de agregación se pretende dar una nueva de forma de representar un conjunto de datos en una única salida. De esta forma, se define primeramente una medida a través de la cual se calculará la forma en la que cada elemento del conjunto de datos tendrá reelevancia en la agregación final. Para ello definimos el concepto de medida difusa.
\begin{definition}\label{def:medidadifusa}
Dado $U$ un universo finito; ${\mathcal{P}(U)}$ el conjunto de todos los subcojuntos de $U$. Una medida difusa es una función ${\mu:\mathcal{P}(U)\rightarrow\unitinterval}$ que satisface que:
\begin{enumerate}
	\item $\mu(\emptyset)=0$ y $\mu(U)=1$.
	\item $A\subseteq B \Rightarrow \mu(A)\leq\mu (B), \forall A, B \subseteq U$.
\end{enumerate}
\end{definition}

A continuación definimos la función integral Choquet conociendo que tomaremos su versión discreta por el contexto en el que se está trabajando.

\begin{definition}\label{def:choquet}
Dado un vector ${(x_1,\dots,x_n)}$, sea $\sigma$ una permutación de ${\{1,...,n\}}$ tal que $x_{\sigma(j)}$ es el $j$-ésimo mayor elemento del vector ${(x_{1},\dots,x_{n})}$. La integral discreta de Choquet con respecto a la medida difusa $\mu$ es
$$Ch_{\mu}(x)=\sum_{i=1}^{n}x_{\sigma(i)}(\mu(\{\sigma(i),\dots,\sigma(n)\})-\mu(\{\sigma(i+1),\dots,\sigma(n)\}))$$
tomando la convención de que ${\{\sigma(n+1),\sigma(n)\}=\emptyset}$.
\end{definition}

\begin{proposition}\label{prop:choque2owa}
Si denotamos a ${w_{\sigma, i}^{\mu} = \mu(\{\sigma(i),\dots,\sigma(n)\})-\mu(\{\sigma(i+1),\dots,\sigma(n)\})}$ se obtiene la siguiente definición de la integral Choquet en función de los operadores OWA definidos en \ref{def:owa}:
$$\sum_{i=1}^{n} w_{\sigma, i}^{\mu} \cdot x_{\sigma(i)}.$$
\end{proposition}

\end{document}
