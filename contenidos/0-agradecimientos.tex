\documentclass[main]{subfiles}

\begin{document}

\dedication{%
Gracias a Javier y a Humberto. Ambos han conseguido que este proyecto avanzase y que, aun con situaciones de desánimo total, siguiera ahí, dándolo todo, para llegar al final. Sin duda alguna, si algo me han demostrado es que la investigación no es nada fácil, y que las alegrías se sirven en frascos pequeños en este mundo.

Gracias a todos mis compañeros, a Luis, a Álvaro, a Kevin, a Iñaki, a Paul, al otro Álvaro, etc. Soy una persona que puedo ser un poco singular en mi forma de trabajar, pero tended por seguro que sea cual haya sido mi acción nunca habrá sido con maldad. Y sí, lo sé, quizá me debería centrar más en lo importante.

Gracias al grupo que formamos Edu, Alfon, Marcos, Pedro y alguno más. Fueron muchas horas juntos y muchas idas de olla. Muchos momentos que recordar y mucha \#EuforiaColectiva que no olvidar.

Gracias a todos mis compañeros de la Comunidad Arrupe y Universitaria del Colegio San Ignacio - Jesuitas de Pamplona. El poder estar con los chavales es un lujo, pero el ver cómo avanza algo que casi vi nacer (por lo menos como es hoy en día) y ver que es imparable, que se sigue poniendo todo el esfuerzo y todas las ganas es maravilloso. Y poder apoyarse en personas como Nacho, Nuño o Maite es estupendo.

Gracias a todos los amigos que he hecho durante estos años por toda España. Alex, Uxue, Dani y todos los demás, porque la lista es muy larga. Es un placer poder decir que participo en RITSI y poder decir lo mucho que me ha aportado, aunque no niego que me haya traido, también, dolores de cabeza. Sin duda, el poder conocer a tanta gente por todo el país es algo que no cambio por nada.

Gracias a mis padres, Juan y Lourdes. Ellos tienen bastante culpa de que hoy esté donde estoy, de que hoy sea quien soy, simplemente de que todo haya funcionado y funcione en mi vida. Y a mi hermano, que siempre aparece de la forma más inoportuna, pero no por ello deja de ser alguien importante y en el fondo, nos queremos.

Gracias a todos aquellos que en mayor o menor medida me he topado en estos cuatro años de universidad, sean estudiantes, profesores, investigadores, PAS, amigos o incluso \enquote{enemigos}.

Este trabajo pone fin a 4 años, 4 años de estudio en la UPNA pero también de muchas más cosas. Si algo tengo ya claro es que no soy el mismo que entró por la puerta el día 2 de septiembre de 2011. No puedo serlo. Y es que estos años han dado para mucho y para mucha gente. Son años que no volverán pero que no se olvidarán.

\blockquote[Steve Jobs, discurso de Standford, 2005]{No hay ninguna razón para no seguir su corazón. [\dots] Tu tiempo es limitado, no lo desperdicies viviendo la vida de otros}
}

\end{document}