\documentclass[main]{subfiles}

\begin{document}

% FUNCIONES DE DOMBI
\section{Funciones de Dombi}\label{sec:dombi}
Las funciones de Dombi \cite{art:dombi} son operadores de equivalencia con una definición diferente a las REF presentadas en la sección \ref{sec:ref}. Con esta relación lo que se pretende conocer es como de iguales son dos elementos de un conjunto difuso dado, pero aplicando nuevas ideas para la construcción del resultado.
%\REV{Definición completa?} Pues así se va a quedar.
\begin{definition}\label{def:dombi}
Dados ${x=(x_1, x_2, \dots,x_n)}$ y ${w=(w_1,w_2,\dots,w_n)}$, denotaremos $D$ como una función de equivalencia de Dombi cuando tengamos que
$$D(w,x)=\frac{1}{2}\left(1+\prod(1-2x_{i})^{w_{i}}\right)$$
\end{definition}
\begin{lemma}\label{def:propiedadesdombi}
La función de equivalencia de Dombi, $D$, cumple las siguientes propiedades:
\begin{enumerate}
	\item $D:\unitinterval\times\unitspace \text{ es continua}$;
	\item $D((w_1,w_2),(0,0)) = 1$;\quad$D((w_1,w_2),(1,1)) = 1$;
	\item $D((w_1,w_2),(0,1)) = 0$;\quad$D((w_1,w_2),(1,0)) = 0$;
	\item $D((w_1,w_2),(x,c(x))) = 0$.
\end{enumerate}
\end{lemma}
%\REV{propiedades}

\end{document}
