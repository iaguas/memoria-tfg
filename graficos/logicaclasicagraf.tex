%%%%%%%%%%%%%%%%%%%%%%%%%%%%%%%%%%%%%%%%%%%%%%%%%%%%%%%%%%%%%%%
%
% Welcome to Overleaf --- just edit your LaTeX on the left,
% and we'll compile it for you on the right. If you give
% someone the link to this page, they can edit at the same
% time. See the help menu above for more info. Enjoy!
%
% Note: you can export the pdf to see the result at full
% resolution.
%
%%%%%%%%%%%%%%%%%%%%%%%%%%%%%%%%%%%%%%%%%%%%%%%%%%%%%%%%%%%%%%%
% Behaviour of the stator voltage and the RMS-value of the stator
% flux as a function of speed in scalar control.
% Author: Erno Pentzin (2013)
\documentclass{article}
\usepackage{amsmath} % Required for \varPsi below
\usepackage{tikz}
%%%<
\usepackage{verbatim}
\usepackage[active,tightpage]{preview}
\PreviewEnvironment{tikzpicture}
\setlength\PreviewBorder{10pt}%
%%%>
\begin{comment}
:Title: AC drive voltage flux control
:Tags: Electrical engineering;Plotting;Basics
:Author: Erno Pentzin
:Slug: ac-drive-voltage

Behaviour of the stator voltage and the RMS-value of the stator
flux as a function of speed in scalar control. Adapted from the
figure in the handout by Jorma Luomi and Asko Niemenmaa (2011).
\end{comment}
\begin{document}

\begin{tikzpicture}

% horizontal axis
\draw[->] (0,0) -- (6.5,0) node[anchor=north] {\footnotesize altura (m)};
% labels
\draw	(0,0) node[anchor=north] {\footnotesize 0}
		(1,0) node[anchor=north] {\footnotesize 0,5}
		(2,0) node[anchor=north] {\footnotesize 1}
		(3,0) node[anchor=north] {\footnotesize 1,5}
		(4,0) node[anchor=north] {\footnotesize 2}
		(5,0) node[anchor=north] {\footnotesize 2,5};

% vertical axis
\draw[->] (0,0) -- (0,2.5) node[anchor=east] {\footnotesize Pertenencia};
% labels
\draw	(0,0) node[anchor=east] {\footnotesize 0 (falso)}
		(0,2) node[anchor=east] {\footnotesize 1 (verdadero)};

% funcion
\draw[thick] (0,0)--(3.5,0);
\draw[thick] (3.5,0)--(3.5,2);
\draw[thick] (3.5,2)--(6,2);
\draw[dotted] (3.5,2.5)--(3.5,0);
\draw (4.75,2.25) node {Alto}; %label
\draw (1.75,2.25) node {No alto}; %label


\end{tikzpicture}

\end{document}